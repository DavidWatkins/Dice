\begin{homeworkProblem}
	\chapter{Types}
	\section{Primitive Data Types:}
	\begin{itemize}
		\item Integer
		\item Double
		\item Void
		\item Character
		\item Boolean
	\end{itemize}
	
	\section{Integer}
	The integer type stores the given value in 32 bits. You should use integer types for storing whole number values (and the char data type for storing characters). The integer type can hold values ranging from -2,147,483,648 to 2,147,483,647.
	
	\section{Double}
	
	The double type stores the given value in 64 bits. You should use double types to store fractional number values or whole number values that do not fit into the range provided by the integer type. The double type can hold values ranging from 1e-37 to 1e37. Since all values are represented in binary, certain floating point values must be approximated. It is therefore recommended that the programmer compare doubles within a given range rather than with the equivalence operator ==.
	
	\section{Void}
	
	The void type is used to indicate an empty return value from a method call. As it is assumed that every method will return a value, and that value must have a type, the type of a return which has no value is null. An example would look like:
	
	public void inc(a) { a++ }
	
	\section{Char}
	
	A character constant is a single character enclosed with single quotation marks, such as ‘p’. The size of the char data type is 8 bits. Some characters cannot be represented using only one character. These extra characters are represented with an “escape sequence”, which consists of a backslash and another character. Some examples are:
	
	\n  Newline character
	\t   Tab character
	\’   Single quotation mark
	\:   Double quotation mark
	
	\section{Bool}
	
	The bool type is a binary indicator which can be set to either True or False. The bool type is stored as one byte (??). A bool must be given a value at the time of declaration:
	
	bool x = True;  // Valid declaration
	bool y;  // Invalid declaration, value not declared
	
	
	
	


\end{homeworkProblem}