\begin{homeworkProblem}
	\chapter{Program Structure and Scope}
	
	Program structure and scope define what variables are accessible and where. When inside a class, there are many different cases of scope, however those are better defined in chapter 7. 
	
	\section{Program Structure}
		
	A Dice program may exist either within one source file or spread among multiple files which can be linked at compile-time. An example of such a linked file is the standard library, or $stdlib.di$. When an include statement is executed at compile time, it will load in the files mentioned at the includes and insert the code at that location as if it were part of the head source file. Therefore at compilation, one only needs to compile with $dicec master.di$. A program consists of zero or more include statements, followed by one or more class definitions. Only one class out of all classes may have a main method, defined with $public void main(char[,] args)$ which designates the entry point for a program to begin executing code. All Dice files are expected to end with the file extension $.di$ and follow the following syntactic layout. 
	
	\begin{minted}{java}
	include(stdlib)
	include(mylib)
	
	class FOO {
	
		(* my code *)
	
	}
	
	class BAR {
	
		(* my code *)
		
		public static void main(char[,] args)
	
	}
	\end{minted}

	
	\section{Scope}
		
		
	Scope refers to which variables, methods, and classes are available at any given time in the program. All classes are available to all other classes regardless of their relative position in a program or library. Variable scope falls into two categories: fields (instance variables) which are defined at the top of a class, and local variables, which are defined within a method. Fields can be public or private. If a field is public then it is accessible whenever an instance of that class is instantiated. For instance, if I have a class X, then class Y can be defined as follows:
	
	\begin{minted}{java}
	class Y {
		
		public int num;
		
		constructor() {
			
			X myObj = X();
			this.num = myObj.number;
		}
	}
	
	class X {
		
		public int number;
		
	}
	\end{minted}
	
	In this example, class Y has one field which is an int. In its constructor, an instance of class X is declared, and a public field within that object is used to set the value for the given int. If a field is declared private, however, it can only be accessed by the methods in the same class. For example, if there is a class Y with a private field, the following is valid:
	
	\begin{minted}{java}
	class Y {
		
		private int num;
		
		constructor() {
			
			this.num = 5;
		}
		
		private int getNum() {
			
			return this.num;
		}
	}
	\end{minted}
		
		However, if I have a class X, that class cannot access the private field within Y. The following is invalid:
	\begin{minted}{java}
	class X {
	
		public int number;
	
		constructor() {
	
			Y myObj = Y();
			(* This code is invalid since num is a private field within Y *)
			this.number = myObj.num;  
		}
	}
	\end{minted}

	Methods are also declared as public or private, and their accessibility is the same as fields. They must have a scope defined on them. \\
	
	Local variables are variables that are declared inside of a method. Local variables are only accessible within the same method in which they are declared, and they may have the same name as fields within the same class since fields in a class are only accessible by calling the $this$ keyword.\\
		
\end{homeworkProblem}