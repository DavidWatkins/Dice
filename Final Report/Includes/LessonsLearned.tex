\begin{homeworkProblem}
	\chapter{Lessons Learned}
	\section{David}
	Most critically I learned that if you want to make something good, put as much effort as physically possible into it. I was told frequently "get started early" with respect to this project. After starting early I also learned that working often and with purpose helped not only myself get through the project but also the rest of my team. 
	
	As project manager the most critical decision I made was to gain consensus on the development environment that each team member was using. My main takeaway was to make sure that everyone agrees to use the same tools and systems. Having incompatible hardware/software can create unnecessary tension in what is already a stressful situation. 
	
	One final note is that I really did not know what to expect from OCaml coming into this class. It seemed very mysterious at first, but after looking through previous examples of compilers from other groups and writing out the Analyzer for my language, I quickly grew to enjoy the language. It certainly was not as daunting as it seemed at first. 
	\section{Emily}
	\section{Khaled}
	Read the lessons learned from previous projects and prioritize (with your group) which of them you will implement. You will not be able to do them all, but if you can agree as a group on which mistakes you can avoid, you're already ahead. For our group, we determined that we will ACTUALLY start early, which we we did.
	
	Fortunately, we had a very organized and decisive manager that made sure we were all on track throughout the semester. Make sure you nominate a person with same qualities if you don't want to spend the last week of the semester pulling all-nighters for this project (save that for your other exams).
	
	Track tasks with Github's issue tracking. Keep this issue tracker open during meetings with the Professor/TAs in order to avoid forgetting discussed to-do items. Ensure the manager of the group delegates through this system.
	
	To spare your team members pain, don't use the diff command's output in your test script. Just label the program's output and your expected output and place them on top of each other for easy reading. 
	\section{Phillip}
\end{homeworkProblem}