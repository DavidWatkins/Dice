\begin{homeworkProblem}
	\section{Introduction}
	Dice is a general purpose, object-oriented programming language. The principal is simplicity, pulling many themes of the language from Java. Dice is a high level language that utilizes LLVM IR to abstract away hardware implementation of code. Utilizing the LLVM as a backend allows for automatic garbage collection of variables as well. \\
	Dice is a strongly typed programming language, meaning that at compile time the language will be type-checked, thus preventing runtime errors of type. \\
	This language reference manual is organized as follows:\\
	\begin{itemize}
		\item Chapter 2 Describes types, values, and variables, subdivided into primitive types and reference types
		\item Chapter 3 Describes the lexical structure of Dice, based on Java. The language is written in the ASCII character set
		\item Chapter 4 Describes the expressions and operators that are available to be used in the language
		\item Chapter 5 Describes different statements and how to invoke them
		\item Chapter 6 Describes the structure of a program and how to determine scope
		\item Chapter 7 Describes classes, how they are defined, fields of classes or their variables, and their methods
		\item Chapter 8 Discusses the different library classes provided with the compiler and their definitions
	\end{itemize}
	The syntax of the language is meant to be reminescent of Java, thereby allowing ease of use for the programmer. 
\end{homeworkProblem}
