\begin{homeworkProblem}
	\section{Classes}
	
	Classes are the constructs whereby a programmer defines their own types. All state changes in a Dice program must happen in the context of changes in state maintained by an object that is an instance of a user-defined class.
	
	\subsection{Class definition}
	A class definition starts with the keyword 'class' followed by the class name (see identifiers in chapter 2) and the class body. The class body, enclosed by a pair of curly braces, declares one or more of each of the following: fields, methods, and constructors.
	
	The members of a class type are all of the following:
	\begin{itemize}
		\item Members inherited from its ancestors (its direct superclass and its ancestors)
		\item Members declared in the body of the class, with the exception of constructors
	\end{itemize}
	
	\subsubsection{Access modifiers}
	Class member declarations must include access modifiers but the class declaration itself does not; there is no notion of a private class in Dice. Field and method declarations must include one of the access modifiers: $public$ or $private$. Fields and methods with the access modifier $public$ can be accessed by methods defined in any class. Fields and methods with the access modifier $private$ can be accessed by methods defined either in the same class or in successor classes (classes derived directly from that class and their successors).
	
	\subsubsection{Fields}
	The only fields that can be declared are instance variables, which are freshly incarnated for each instance of the class.
	Field declarations have the following format:
	\begin{minted}{java}
	<access modifier> <type> <VariableDeclaratorId>;
	(* Example *) private int myInstanceVariable;
	\end{minted}

	All instance variables must be declared before methods and constructors. 
	
	\subsubsection{Methods}
	A method declares executable code that can be invoked, passing a fixed number of values as arguments. The only methods that can be declared are the 'main' method and instance methods. Instance methods are invoked with respect to some particular object that is an instance of a class type.\\
	
	Method declarations constitute a method header followed by a method body. The method header has the following format:
	\begin{minted}{java}
	<access modifier> <return type> <method name> <comma-separated list of parameters>
	(* Example *) public double amountPaid(double wage, int duration)
	\end{minted}
	
	The method body contains, enclosed between the ASCII characters '\textbraceleft' and '\textbraceright', zero or more variable declarations followed by zero or more statements. If the type of the return value is not void, then the method body must include a return statement.\\
	
	One and only one of the classes to be compiled must contain a definition for a method named "main" that executes when the program runs. The $main$ method is not callable as an instance method. The $main$ method must have a void return type and accept a single parameter of type char[][]. Hence, its signature must be:
	\begin{minted}{java}
	public void main (char[][] args)
	\end{minted}
	
	If either zero or more than one class contains a definition for a method with the signature above, this results in a compile-time error.\\
	
	Methods can be overloaded: If two methods of a class (whether both declared in the same class, or both inherited by a class, or one declared and one inherited) have the same name but signatures that are not equivalent, then the method name is said to be overloaded. There can be multiple methods with the same name defined for a class, as long as each has a different number and/or type of parameters. The $main$ method can never be overloaded because it has one and only one accepted signature. If two methods in the same class have the same signature, the compiler throws an error.\\
	
	\subsubsection{Constructors}
	Constructors are similar to methods but cannot be invoked as an instance method; they are used to initialize new class instances. A constructor has no return type and its formal parameters are identical in syntax and semantics to those of a method. A constructor definition has the following format:
	\begin{minted}{java}
	constructor (<comma-separated formal arguments>) {
		<list of variable declarations>
		<list of statements>
	}
	(* Example *) constructor (int a, char[] b) {...}
	\end{minted}

	Unlike fields and methods, access to constructors is not governed by access modifiers. Constructors are accessible from any class.\\
	
	Constructor declarations are never inherited and therefore are not subject to overriding.\\
	
	If no constructors are defined, the compiler defines a default constructor. Like methods, they may be overloaded. It is a compile-time error to declare two constructors with equivalent signatures in a class.\\
	
	When the programmer declares an instance of the class, either a user-defined constructor or the default constructor is automatically called.
	\begin{minted}{java}
	class Foo {
		constructor (int x) {...}
		...
	}
	class Bar {
		public void main (char[][] args) {
			int x; 
			Foo myFooObj;
			x = 5;
			myFooObj = Foo(x);
		}
	}
	\end{minted}
	
	\subsection{Referencing instances}
	The keyword 'this' is used in the body of method and constructor declarations to reference the instance of the object that the method or constructor will bind to at runtime.
	
	\subsection{Inheritance}
	The members of a class include both declared and inherited members. A class inherits all members of its direct superclass and superclasses of that class. To define a class $Y$ that inherits members of an existing class named "X" and all superclasses of $X$, use the keyword $extends$ when defining $Y$.
	\begin{minted}{java}
	class Y extends X {...}
	\end{minted}	
	\subsubsection{Overriding}
	Newly declared methods can override methods declared in any ancestor class. An instance method m1, declared in class C, overrides another instance method m2, declared in class A iff both of the following are true:
	\begin{itemize}
		\item C is a subclass of A
		\item The signature of m1 is identical to the signature of m2
	\end{itemize}
	
\end{homeworkProblem}
