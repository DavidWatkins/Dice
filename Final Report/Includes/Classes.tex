\begin{homeworkProblem}
	\section{Classes}
	
	Classes are the constructs whereby a programmer defines their own types. All state changes in a Dice program must happen in the context of changes in state maintained by an object that is an instance of a user-defined class.
	
	\subsection{Class Declaration}
	The syntax for declaring a class is in the "Declarations" subsection of the "Statements" section. According to the class declaration syntax, fields, constructor and methods are optional for each class and may appear in any order in the class body. 
	\newline
	Methods may not be overloaded: For any method name, only one method per class may be defined with that name.
	\newline
	If no constructors are defined, the compiler defines a default constructor. Unlike methods, they may be overloaded. When the programmer declares an instance of the class, either a user-defined constructor or the default constructor is automatically called. It is a compile-time error to declare two constructors with equivalent signatures in a class.
	
	\subsection{Inheritance}
	Dice supports multiple levels of inheritance. The syntax for declaring a class that inherits from another class via the \textbf{extends} keyword is in the "Declarations" subsection of the "Statements" section. A class inherits the public fields and methods of all its ancestors. Constructors are not inherited.
	
	\subsubsection{Overriding}
	A class can override any inherited method by defining its own method with the same method signature and a custom body. Two method signatures are considered to be the same if they match on their return type and name and have the same number of formal arguments, with the sequence of types of their formals matching. Constructor declarations are never inherited and therefore are not subject to overriding.
	
	\subsection{Access Modifiers}
	Fields and methods must have one of the following access modifiers: \textbf{public} $|$ \textbf{private}. If a field or method has a public access modifier, then it may be accessed by the method of any class in the program. Private fields and methods are accessible from within the class in which they are declared, but not from any descendant classes.
	\newline
	Unlike fields and methods, access to constructors is not governed by access modifiers. Constructors are accessible from any class.
	
	
	\subsection{Referencing instances}
	The keyword \textbf{this} is used in the body of method and constructor declarations to reference the instance of the object that the method or constructor will bind to at runtime.
	
\end{homeworkProblem}
