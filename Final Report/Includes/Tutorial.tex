\begin{homeworkProblem}
	\chapter{Language Tutorial}
	\section{Environment Setup}
	The compiler has been built an tested using an Ubuntu 15.10 virtual machine. The ISO for downloading Ubuntu 15.10 can be found here\footnote{http://www.ubuntu.com/}. This is followed by downloading virtualbox and following the corresponding tutorial for setting up a custom Ubuntu VM here\footnote{http://www.wikihow.com/Install-Ubuntu-on-VirtualBox}. 
	Once inside the VM there are a series of packages that need to be installed before you can compile the compiler. Run the following commands to install the corresponding packages:
	\begin{minted}[breaklines]{bash}
>sudo apt-get install m4 clang-3.7 clang-3.7-doc libclang-common-3.7-dev libclang-3.7-dev libclang1-3.7 libclang1-3.7-dbg libllvm-3.7-ocaml-dev libllvm3.7 libllvm3.7-dbg lldb-3.7 llvm-3.7 llvm-3.7-dev llvm-3.7-doc llvm-3.7-examples llvm-3.7-runtime clang-modernize-3.7 clang-format-3.7 python-clang-3.7 lldb-3.7-dev liblldb-3.7-dbg opam llvm-runtime
	\end{minted}
	copy the tutorial from https://github.com/DavidWatkins/Dice
	\section{Using the Compiler}
	Inside the directory 'Dice' type \textbf{make}. This creates the dice compiler that takes in '.dice' files and compiles them to corresponding '.ll' files corresponding to LLVM IR. The syntax for running the dice executable is: \textbf{dice [optional-option] $\langle$source file$\rangle$}. There are also additional flags with respect to the compiler that allow for additional options.
	\begin{itemize}
		\item \textbf{-h} - Print help text
		\item \textbf{-tendl} - Prints tokens with newlines intact
		\item \textbf{-t} - Prints token stream
		\item \textbf{-p} - Pretty prints Ast as a program
		\item \textbf{-ast} - Prints abstract syntax tree as json
		\item \textbf{-sast} - Prints semantically checked syntax tree as json
		\item \textbf{-c} - Compiles source and prints result to stdout
		\item \textbf{-f} - Compiles source to file ($\langle$filename$\rangle$.$\langle$ext$\rangle$ $\rightarrow$ $\langle$filename$\rangle$.ll)
	\end{itemize}
	
	The following sample dice code demonstrates the following features:
	\begin{itemize}
		\item The mandatory main function that exists within \textbf{only} one class. The syntax for a main declaration is \textbf{public void main(char[][] args)}
		\item Calling the built-in print function, which takes an arbitrary list of primitive values, including char[]. 
		\item A string literal with escape characters
		\item Defining a base class with one or more fields. 
	\end{itemize}
	\begin{minted}[breaklines,linenos]{java}
	class example1 {
		public void main(char[][] args) {
			print("This is example 1\n");
		}
	}
	\end{minted}
	To compile the sample code above, type:
	\begin{minted}{bash}
	> ./dice example1.dice
	\end{minted}
	The output will be:
	\begin{minted}{bash}
	>lli example1.ll
	This is example 1
	>
	\end{minted}
	\section{Defining methods}
	\section{Control Flow}
	\section{Defining custom classes}
	\section{Using Inheritance}
\end{homeworkProblem}