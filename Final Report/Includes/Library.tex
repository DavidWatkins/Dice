\begin{homeworkProblem}
	\section{Built in Functions}
	\section{Standard Library Classes}
	\subsection{Accessing the Standard Library}
	To access the standard library, enter 'include(stdlib);' at the top of the source code. As noted earlier, including a file can only occur once, so do not include a class a second time. 
	
	\subsection{String}
	Dice provides certain standard library classes to assist the user with string manipulation and file I/O.

	\subsubsection{Fields}
	String has no public fields

	\subsubsection{Constructors}
	\paragraph{String(char[] a)}
	Accepts a char array, such as a string literal or a char array, and creates a String object

	\subsubsection{Methods}
	\paragraph{public bool contains(char[] chrs)}
	Returns true if and only if this string contains the specified sequence of char values.
	\paragraph{public int indexOf(int ch)}
	Returns the index within this string of the first occurrence of the specified character.
	\paragraph{public bool isEmpty()}
	Returns true if and only if length() is 0.
	\paragraph{public int length()}
	Returns the length of the string.
	\paragraph{public char[] toCharArray()}
	Returns the char array of this string.

	\subsection{File}
	The File class constructor takes one argument which is a char[] that points to a file on which the user wishes to operate. The constructor stores the given path in a field and then calls open() on the given path and, if successful, sets the object’s file descriptor field to the return of open(). If open() fails, the program exits with error.
	\subsubsection{Fields}
	File has no public fields

	\subsubsection{Constructors}
	\paragraph{File(char[] path, bool isWriteEnabled)}
	Accepts a char array to open a file on, then creates a file object with the file descriptor. isWriteEnabled is a parameter that is used to determine whether the file can be written to or just read from.

	\subsubsection{Methods}
	\paragraph{public char[] read(int num)}
	Reads num bytes from the open file and returns the bytes in a char array.
	\paragraph{public void close()}
	Closes the open file. On error, the program exits with error.
	\paragraph{public void write(char[] arr)}
	Writes the contents of the char[] array to the file

\end{homeworkProblem}
