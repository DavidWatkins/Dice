\begin{homeworkProblem}
	\chapter{Introduction}
	The Dice programming language is an object-oriented, general purpose programming language. It is designed to let programmers who are more familiar with object oriented programming languages to feel comfortable with common design patterns to build useful applications. The syntax of Dice resembles the Java programming language. Dice compiles down to LLVM IR which is a cross-platform runtime environment. This allows Dice code to work on any system as long as there is an LLVM port for it, which includes Windows, Mac OS X, and Linux or various processor architectures such as x86, MIPS, and ARM\footnote{http://llvm.org/}.\\
	
	Dice lays programs out the same way a Java program would. Variables and methods of a class can be declared with private scope. There is a simple to use inheritance that allows for multiple children inheriting the fields and methods of its parent. Dice also allows for convenient use of functions that exist in C, such as malloc, open, and write. This allows the user to construct objects and call c functions using those objects. 
	
	\section{Background}
	Object-oriented programming (OOP) is a programming paradigm based on the concept of "objects". These objects are data structures that contain data, in the form of fields, often known as attributes. The code itself are contained within methods in the code which are compiled to varying subroutines. The most useful aspect of OOP is that these methods and fields can modify one another allowing for a rich and varied use case. \\
	
	Class based OOP specifically creates instances of classes, referred to as objects, which have their values modified at runtime. There are many languages that implement their language this way including Java and C\#. \\
	
	Inheritance is when an object or class is based on another class using the same implementation. This allows for a class to serve as a blueprint for subclasses. Polymorphism allows an object to take on many forms. This may include an object being assigned to a type that is a class it inherits from, or being used in place of a class it inherits from. \\
	
	We want to leverage these capabilities using LLVM code to produce a syntactically Java-like language but offer a cross platform solution that is simple and easy to use. Implementing inheritance and objects in a c-like context like LLVM allows for fine control over the code. \\
	
	\section{Related Work}
	Object-oriented programming languages have existed since the late 20th century. Java, C\#, C++, Objective-C, Python, and many more languages have facilities for defining custom user classes and manipulating them at runtime. \\
	
	Implementing an object-oriented paradigm using C is a well-known solution, but compiling object-oriented code down to LLVM is not publicly available. We want to contribute to the LLVM community by adding additional information regarding the creation of a compiler using OCaml that compiles to LLVM code. 
	
	\section{Goals}
	\subsection{Cross-Platform}
	Utilizing the LLVM IR we are able to compile the source once and have it work on multiple architectures without fail.\\
	\subsection{Flexibility}
	Allowing the user to define their own classes and offering them the ability to inherit functionality from other user defined types offer a wide range of possibilities for their programs and also saves the user time when implementing large programs.\\
	\subsection{Transparency}
	Using the LLVM IR allows the user to see exactly what the program is doing after the compiler is done. For a more optimal result it can then be compiled to bitcode representation using the LLVM compiler. \\
	\subsection{Familiarity}
	Incorporate familiar primitive data types most commonly found in languages such as C, C++, and Java such as int, char, float, and bool. 
\end{homeworkProblem}