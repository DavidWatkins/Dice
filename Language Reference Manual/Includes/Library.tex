\begin{homeworkProblem}
    \chapter{Supplementary Code}
	\section{Standard Library}
    The standard library was written in order to provide the user with a solid foundation on which to start writing interesting programs. To that end we provide for basic file i/o and string and integer manipulation.
	
	\section{String}
    Proivde useful functionality for string manipulation.

	\subsection{Fields}
	String has no public fields. Private fields include a char array my_string which stores the given string and an int to store the length of the string. 

	\subsection{Constructors}
	\subsubsection{String(char[] a)}
    Accepts a char array, such as a string literal or a char array. This string is copied into the my_string field of the object and the private length() method is run to get the length of the input string.

	\subsection{Methods}
    \subsubsection{private int length_internal(char[] input)}
    Returns the length of the given char array.
    \subsubsection{private char[] copy_iternal(char[] input)}
    Creates a new char array into which it copies the given char array.
    \subsubsection{public char[] string()}
    Returns the char array contained in the my_string field.
    \subsubsection{public char getChar(int index)}
    Returns the char containted at the given index in the my_string field.
    \subsubsection{public int length()}
    Returns the length of the my_string field
    \subsubsection{public int toInteger()}
    Converts the char array in the my_string field to an integer and returns that int. If the char array contained in the my_string field is not a  string representation of an int, the behavior is undefined.
    \subsubsection{public int toDigit(char digit)}
    Returns the integer corresponding to the character passed in.
    \subsubsection{public class String copy(class String input)}
    Returns a copy of the current object.
    \subsubsection{public int indexOf(char input)}
    Returns the index of the input character in the my_string field. Returns -1 if the character is not found in the field.
    \subsubsection{public class String reverse()}
    Returns a string object with the my_string field containting the reverse of the current my_string char array.
    \subsubsection{public class String concat(class String temp)}
    Returns a string object with the my_string field containing the concatenation of the current my_string field with the temp's my_string field.
    \subsubsection{public bool compare(class String input)}
    Returns true if the my_string field of the input String is equal to the my_string field of the current String object.
    \subsubsection{public bool contains(class String check)}
    Returns true if the my_string field of the input String is contained in the my_string field of the current String object.
    \subsubsection{public void free()}
    Frees the memory for the my_string field of the current String object.

	\section{File}
	The File class constructor takes two arguments: a char[] that points to an already opened file on which the user wishes to operate and a boolean indicating whether the user wishes to open the file for writing. If the boolean is true the file is opened for reading and writing, and if false the file is opened as read only. The constructor stores the given path in a field and then calls open() on the given path and, if successful, sets the object’s file descriptor field to the return of open(). If open() fails, the program exits with error.
	\subsection{Fields}
	File has no public fields. Private fields are the class String filePath, private bool isWriteEnabled, and the private int fd.

	\subsection{Constructors}
	\subsubsection{File(char[] path, bool isWriteEnabled)}
	Accepts a char array to open a file on, then creates a file object with the file descriptor. isWriteEnabled is a parameter that is used to determine whether the file can be written to or just read from.

	\subsection{Methods}
    \subsubsection{private int openfile(class String path, bool isWriteEnabled)}
    Returns the file descriptor of the opened file if successful, and -1 otherwise. 
	\subsubsection{public char[] readfile(int num)}
	Reads num bytes from the open file and returns the bytes in a char array.
	\subsubsection{public int writefile(char[] arr, int offset)}
	Writes the contents of the char[] array to the file. If offset is -1 the write starts at the beginning of the file, if 0 it starts at the end of the file, and with any other positive integer it starts writing offset bytes from the beginning of the file.
	\subsubsection{public void closefile()}
	Closes the open file. On error, the program exits with error.

    \section{Integer}
    The Integer class provides for integers to be converted to char arrays.
    \subsection{Fields}
    Integer has no public fields. There is one private field my_int which stores the given integer.

    \subsection{Constructors}
    \subsubsection{Integer(int input)}
    Accepts an integer which is stored in the field my_int.

    \subsection{Methods}
    \subsubsection{public int num()}
    Returns the integer stored in the my_int field.
    \subsubsection{public char toChar(int digit)}
    Returns in teh input digit as a character.
    \subsubsection{public class String toString()}
    Converts the integer stored in the my_int field into a string using the toChar() method. Returns a string object.
\end{homeworkProblem}
