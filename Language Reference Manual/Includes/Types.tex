\begin{homeworkProblem}
	\chapter{Types}
	\section{Primitive Data Types}

	\subsection{int}
	The integer type stores the given value in 32 bits. You should use integer types for storing whole number values (and the char data type for storing characters). The integer type can hold values ranging from -2,147,483,648 to 2,147,483,647.

	Syntactically correct use of int:

	\begin{minted}{java}
	<scope> int funcName( <formal-opts> ) { <body> }
	<scope> <type> funcName( <formal-opts> ) { int i; <body> }
	class className {
		<scope> int i;
		<cbody>
	}
	\end{minted}


	\subsection{float}

	The float type stores the given value in 64 bits. You should use float types to store fractional number values or whole number values that do not fit into the range provided by the integer type. The float type can hold values ranging from 1e-37 to 1e37. Since all values are represented in binary, certain floating point values must be approximated. \\

	Syntactically correct use of float:

	\begin{minted}{java}
	<scope> float funcName( <formal-opts> ) { <body> }
	<scope> <type> funcName( <formal-opts> ) { float i; <body> }
	class className {
		<scope> float i;
		<cbody>
	}
	\end{minted}


	\subsection{void}

	The void type is used to indicate an empty return value from a method call. As it is assumed that every method will return a value, and that value must have a type, the type of a return which has no value is null. A void type cannot be used for a variable type, only function return types. An example would look like: "public void inc(a) { a = a + 1; }" \\

	Syntactically correct use of void:


\begin{minted}{java}
		<scope> void funcName( <formal-opts> ) { <body> }
	\end{minted}


	\subsection{char}

	A character constant is a single character enclosed with single quotation marks, such as 'p'. The size of the char data type is 8 bits or 1 byte. Some characters cannot be represented using only one character.\\

	Syntactically correct use of char:


	\begin{minted}{java}
	<scope> char[] funcName( <formal-opts> ) { <body> }
	<scope> <type> funcName( <formal-opts> ) { char i; <body> }
	class className {
		<scope> char i;
		<cbody>
	}
	\end{minted}


	\subsection{bool}

	The bool type is a binary indicator which can be set to either True or False. A bool can take one of two values, 'true' or 'false'. A bool could also be null.\\

	Syntactically correct use of bool:

	\begin{minted}{java}	
	class className {
		<scope> bool i;
		<cbody>
	}
	\end{minted}
	or
	\begin{minted}{java}
	<scope> bool funcName( <formal-opts> ) { <body> }
	private bool foo() {
		bool i;
		i = true;
		return i;
	}
	\end{minted}
	\section{Non-Primitive Data Types}
	\subsection{Arrays}
	An array is a data structure which lets you store one or more elements consecutively in memory. Elements in an array are indexed beginning at position zero, not one.\\

	You declare an array by specifying the type of its elements, its name, and its number of dimensions.	
	
	Array Declaration:
	\begin{minted}{java}
	(* single dimension: <type T>[] arrayName; *)
	int[] myArray;

	(* multi dimension: <type T>[][]...[] arrayName; *)
	float[][][] myThreeDimensionalArray;
	\end{minted}
	
	Initialization of the array or its elements must be executed in a statement following the array declaration.
	Single dimension:
	\begin{minted}{java}
	<type T>[] arrayName;
	arrayName = <type T>[number elements];
	(* declaration *)
	float[] a;
	(* initialize to an array of 8 float elements *)
	a = float[8];
	(* alternatively initialize each element in the array *)
	a = |1.1, 2.2, 3.3, 4.4, 5.5, 6.6, 7.7, 8.8|;
	\end{minted}

	Multidimension:
	\begin{minted}{java}
	<type T>[]...[] arrayName;
	arrayName = <type T>[num elements in outermost dim]...[num elements in innermost dim];
	int[][] b;
	(* assigns an array of size 4x1 into b *)
	b = int[4][1];
	\end{minted}

	Executing this statement after b has already been initialized as above overwrites the value at b with a new array of the same size but with assigned values for each element:
	\begin{minted}{java}
	b = ||1|,|2|,|3|,|4||;
	\end{minted}

	These are all legal array initializations:
	\begin{minted}{java}
	int[][] b;
	b = ||1|,|2|,|3|,|4||;

	int[][] b;
	b = int[4][1];
	b[0][0] = 1;
	b[1][0] = 2;
	\end{minted}

	The following statements declare and initialize a two dimensional char array, since a string literal evaluates to a single-dimension char array.
	\begin{minted}{java}
	char[][] a;
	a = |"Hello World"|;
	\end{minted}

	The following statements result in a compile-time error:
	\begin{minted}{java}
	(* Cannot declare and initialize in same statement *)
	int[] myArray = |3, 4|;
	(* Cannot specify array length in declaration *)
	int[2] myArray;
	\end{minted}

	The syntax for array access is: arrayName[indexToAccess];	
	\begin{minted}{java}
	int[] a;
	a = |1,2,3|;
	a[2]; (* 3 *)
	a[2] = 5; (* change the value at the specified index to 5 *)
	a[2]; (* 5 *)
	\end{minted}

	The type of an array can be any primitive or an array type. This means that you can declare an n-dimensional array, the members of which can be accessed by first indexing to the desired element of the outermost array, which is of type array, and then accessing into the desired element of the immediately nested array, and continuing n-1 times.

	\begin{minted}{java}
	char[][] a;
	a = |"Hello World"|;
	a[0][6]; (* 'W' *)
	\end{minted}
	
	Arrays have a `length` method that returns the number of elements contained.
	\begin{minted}{java}
	char[][] a;
	a = |"Hello World"|;
	a.length; (* 1 *)
	a[0].length; (* 11 *)
	\end{minted}
	
	\subsection{Objects}
	
	See chapter 7 on classes to learn more about the syntax and usage of objects.

	\section{Casting}
	Casting is not supported in this language. There are interesting behaviors between ints and float defined in the section on operators that imitate casting, but there is no syntax to support casting between types directly. 

\end{homeworkProblem}
